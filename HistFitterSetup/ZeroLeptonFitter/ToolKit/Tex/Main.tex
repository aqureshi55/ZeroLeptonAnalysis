\documentclass[landscape,12pt,a4paper]{article}
\usepackage[landscape]{geometry}
\usepackage{cite,mcite}
\usepackage{graphicx}
\usepackage{subfigure}
\usepackage{atlasphysics}
\usepackage{atlas_title,ifthen}
\usepackage{helvet}

\def\atlasnote#1{\def\mydocversion{{\large ATL-#1}}}
\def\preprint#1{\def\mydocversion{{\large #1}}}
\def\thedate{\today}

\newlength{\capindent}
\setlength{\capindent}{0.5cm}
\newlength{\capwidth}
\setlength{\capwidth}{\textwidth}
\addtolength{\capwidth}{-2\capindent}
\newlength{\figwidth}
\setlength{\figwidth}{\textwidth}
\addtolength{\figwidth}{-2.0cm}
\newcommand{\icaption}[2][!*!,!]{\hspace*{\capindent}%
  \begin{minipage}{\capwidth}
    \ifthenelse{\equal{#1}{!*!,!}}%
      {\caption{#2}}%
      {\caption[#1]{#2}}
      \vspace*{3mm}
  \end{minipage}}

% Shorthand for \phantom to use in tables
\newcommand{\pho}{\phantom{0}}
\newcommand{\bslash}{\ensuremath{\backslash}}
\newcommand{\BibTeX}{{\sc Bib\TeX}}
% Upsilon(1S)
\newcommand{\UoneS}{\ensuremath{\Upsilon(\mathrm{1S})}}
%

%%%%%%%%%%%%%%%%%%%%%%%%%%%%%%%%%%%%%%%%%%%%%%%%%%%%%%%%%%%%%%%%%%%%%%%%%%%%%%%
% Mes commandes
%%%%%%%%%%%%%%%%%%%%%%%%%%%%%%%%%%%%%%%%%%%%%%%%%%%%%%%%%%%%%%%%%%%%%%%%%%%%%%%
\newcommand{\ourintlumi}{{20.3\ }}
\newcommand{\ppb}{\mbox{\ensuremath{p\bar p}}}
\newcommand{\invpb}{pb$^{-1}$}
\newcommand{\etadet}{\mbox{\ensuremath{\eta_\mathrm{det}}}}
\newcommand{\zdca}{\mbox{\ensuremath{z_\mathrm{DCA}}}}
\newcommand{\rdca}{\mbox{\ensuremath{r_\mathrm{DCA}}}}
\newcommand{\ptgam}{\mbox{\ensuremath{p_{T}^\gamma}}}
\newcommand{\ptZ}{\mbox{\ensuremath{p_{T}^Z}}}
\newcommand{\ptjet}{\mbox{\ensuremath{p_{T}^\mathrm{jet}}}}
\newcommand{\etadetjet}{\mbox{\ensuremath{\eta_\mathrm{det}^\mathrm{jet}}}}
\newcommand{\zee}{\mbox{\ensuremath{Z \to ee }}}
\newcommand{\pythia}{{\sc PYTHIA}}
\def\Ereco{\ensuremath{E^{\mathrm{reco}}}}
\def\Etrue{\ensuremath{E^{\mathrm{true}}}}
\def\Mreco{\ensuremath{M^{\mathrm{reco}}}}
\def\Mtrue{\ensuremath{M^{\mathrm{true}}}}
\def\bias{\mathrm{bias}}
\def\fit{\mathrm{fit}}
\def\inj{\mathrm{inj}}
\def\reco{\mathrm{reco}}
\def\gen{\mathrm{gen}}
\def\true{\mathrm{true}}
\def\NDF{\ensuremath{\mathrm{NDF}}}
%\newcommand{\antikt}{\mbox{\ensuremath{Anti-K_{T}}}}
\newcommand{\antikt}{\mbox{Anti-K\ensuremath{_T}}}

\def\figpath{Plots}

\newdimen\figsize
\setlength\figsize{\hsize}%

\newdimen\fullfigsize
\setlength\fullfigsize{\hsize}%

\long\def\twoboxesgap#1#2#3{%
  \setlength\figsize{\hsize}%
  \addtolength\figsize{-#3}%
  \divide\figsize by 2
  \vbox{%
  \makebox{\parbox[t]{\figsize}{\vskip 0.1pt #1}%
           \hspace{#3}%
           \parbox[t]{\figsize}{\vskip 0.1pt #2}}}}

% This command sets its two arguments in two side-by-side parboxes.
% The widths of the boxes are set to fill the page width, with a
% a gap of 2\columnsep between them.
\long\def\twoboxes#1#2{\twoboxesgap{#1}{#2}{2\columnsep}}


%%%%%%%%%%%%%%%%%%%%%%%%%%%%%%%%%%%%%%%%%%%%%%%%%%%%%%%%%%%%%%%%%%%%%%%%%%%%%%%
% This is where the document really begins
%%%%%%%%%%%%%%%%%%%%%%%%%%%%%%%%%%%%%%%%%%%%%%%%%%%%%%%%%%%%%%%%%%%%%%%%%%%%%%%
%
\begin{document}
\clearpage
\include{yield_SR6jl}
\include{yield_SR6jm}
\include{yield_SR2jt}
\include{yield_SR6jtplus}
\include{yield_SR4jlminus}
\include{yield_SR4jt}
\include{yield_SR2jl}
\include{yield_SR2jm}
\include{yield_SR5j}
\include{yield_SR4jl}
\include{yield_SR4jm}
\include{yield_SR3j}
\include{yield_SR6jt}
\clearpage
\include{systtable_SR6jl}
\include{systtable_SR6jm}
\include{systtable_SR2jt}
\include{systtable_SR6jtplus}
\include{systtable_SR4jlminus}
\include{systtable_SR4jt}
\include{systtable_SR2jl}
\include{systtable_SR2jm}
\include{systtable_SR5j}
\include{systtable_SR4jl}
\include{systtable_SR4jm}
\include{systtable_SR3j}
\include{systtable_SR6jt}
\clearpage
\begin{figure}[H]\begin{center}\begin{tabular}{cc}\includegraphics[width=1\textwidth]{histpull_SR6jl}\end{tabular}\end{center}\end{figure}
\begin{figure}[H]\begin{center}\begin{tabular}{cc}\includegraphics[width=1\textwidth]{histpull_SR6jm}\end{tabular}\end{center}\end{figure}
\begin{figure}[H]\begin{center}\begin{tabular}{cc}\includegraphics[width=1\textwidth]{histpull_SR2jt}\end{tabular}\end{center}\end{figure}
\begin{figure}[H]\begin{center}\begin{tabular}{cc}\includegraphics[width=1\textwidth]{histpull_SR6jtplus}\end{tabular}\end{center}\end{figure}
\begin{figure}[H]\begin{center}\begin{tabular}{cc}\includegraphics[width=1\textwidth]{histpull_SR4jlminus}\end{tabular}\end{center}\end{figure}
\begin{figure}[H]\begin{center}\begin{tabular}{cc}\includegraphics[width=1\textwidth]{histpull_SR4jt}\end{tabular}\end{center}\end{figure}
\begin{figure}[H]\begin{center}\begin{tabular}{cc}\includegraphics[width=1\textwidth]{histpull_SR2jl}\end{tabular}\end{center}\end{figure}
\begin{figure}[H]\begin{center}\begin{tabular}{cc}\includegraphics[width=1\textwidth]{histpull_SR2jm}\end{tabular}\end{center}\end{figure}
\begin{figure}[H]\begin{center}\begin{tabular}{cc}\includegraphics[width=1\textwidth]{histpull_SR5j}\end{tabular}\end{center}\end{figure}
\begin{figure}[H]\begin{center}\begin{tabular}{cc}\includegraphics[width=1\textwidth]{histpull_SR4jl}\end{tabular}\end{center}\end{figure}
\begin{figure}[H]\begin{center}\begin{tabular}{cc}\includegraphics[width=1\textwidth]{histpull_SR4jm}\end{tabular}\end{center}\end{figure}
\begin{figure}[H]\begin{center}\begin{tabular}{cc}\includegraphics[width=1\textwidth]{histpull_SR3j}\end{tabular}\end{center}\end{figure}
\begin{figure}[H]\begin{center}\begin{tabular}{cc}\includegraphics[width=1\textwidth]{histpull_SR6jt}\end{tabular}\end{center}\end{figure}

\end{document}
